\subsection{Justificación}
Respecto a la vivienda inteligente o la domótica en el hogar, se puede decir que la literatura es amplia y que existen diferentes propuestas comerciales de desarrollo como lo son: Calaos, Domoticz, Home Assistant, OpenHAB, OpenMotics, donde cada marca propone su marco de trabajo, y diferentes plataformas de desarrollo tanto software como hardware. Una desventaja de estas propuestas es su interfaz hardware, en la mayoría de los casos el hardware adicional corresponde a dispositivos desarrollados y mantenidos por ellos mismos, lo que significa que limita la escalabilidad de las
\vspace{0.5cm}\\
Las anteriores tecnologías son propuestas donde un centro de mando controla dispositivos externos, pero, por otra parte, si pensamos en cada módulo hardware con la capacidad de conectarse a la red de redes de manera independiente (Iot) como lo hace la propuesta realizada el en proyecto del “Smart switch” \cite{GiralSala2016}; se tiene como ventaja la independencia sobre las características físicas del hogar o de la capacitación técnica del instalador del sistema, pero como desventaja, al momento de escalar el problema a un mayor número de circuitos se pueden presentar problemas de saturación de la red inalámbrica (wifi) y sería más costoso debido al hardware adicional para cada nuevo circuito que se desee controlar.
\vspace{0.5cm}\\
Teniendo en cuenta las comparaciones anteriores, se podría pensar que un sistema que englobe varios sub sistemas podría ser el adecuado desde el punto de vista de la escalabilidad, pero esa no es la única razón; Hoy en día los métodos de generación alternativa y energía renovable se encuentran en crecimiento, por lo que nace la necesidad de interconectar redes eléctricas inteligentes, es decir, una red con un componente de generación y consumo tanto como AC y DC. Este tipo de topología eléctrica se le conoce como nanogrid, y se ampliará la teoría respecto a este concepto en futuras secciones del documento. Esta topología eléctrica añade como variable la generación eléctrica junto con una nueva forma fisica de energía (corriente directa).
\vspace{0.5cm}\\
Inclusive, se puede observar que artículos académicos se han dedicado a analizar este concepto en el contexto Colombiano,  por ejemplo un estudio sobre microgrids mencionó la importancia y necesidad de implementar viviendas inteligentes como se ve a continuación: “desde el contexto de seguridad eléctrica, equidad social y mitigación del impacto ambiental en Colombia, el sistema energético debe afrontar los nuevos retos requeridos para satisfacer la demanda. Desde un punto de vista técnico, es necesario dotar la red tradicional con las características de una red inteligente ”. 
\vspace{0.5cm}\\
Pero ¿acaso una red inteligente implica una casa inteligente? La respuesta es; parcialmente sí. Considerando que hoy en día toda vivienda realiza un monitoreo o medición de al menos 2 variables críticas; el consumo eléctrico, y el consumo de agua, ambos son de interés para la compañía prestadora de servicios públicos en la vivienda.
\vspace{0.5cm}\\
Resaltando la importancia de las anteriores variables físicas y teniendo en cuenta el modelo de nanogrid al cual podría llegar a ser una vivienda; la gestion de informacion adicional a la necesaria para cumplir la domótica resalta al a vista. En adición, estos componentes de generación y consumo eléctrico son la base para hablar sobre el impacto ambiental de las costumbres internas de los habitantes de un hogar, que resulta de vital interés para la optimización de los 4 ítems de calificación en el solar decathlon.
\vspace{0.5cm}\\
Finalizando, una propuesta integral de domótica y gestión inteligente de información como la que se presenta en este documento facilita la verificación y el cumplimiento de los ítems anteriormente mencionados, contemplando que el sistema quedara diseñado de tal manera que sea fácil la escalabilidad para aspectos como; seguridad, aseguramiento contra accidentes (incendio), viviendas de diferente tamaño, interfaz de voz y video.

