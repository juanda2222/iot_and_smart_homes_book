\subsection{Planteamiento del problema}

Hoy en día las tecnologías orientadas internet y la mayoría de los objetos utilizados en la cotidianidad, en general, están cada vez más relacionadas entre sí; el pasado año se vio un incremento de 2.79 billones de nuevos dispositivos conectados a la red de redes (entre los cuales no solo se encuentran celulares computadores y tablets). Nuevos dispositivos como aires acondicionados, televisores y electrodomésticos en general son lanzados al mercado con la capacidad de comunicarse con el usuario a través de aplicaciones web o dispositivos móviles día tras día. Lo anterior se debe al aumento en los desarrollos de hardware y software para sistemas con la capacidad de acceso a la red; es decir, con la capacidad de acceder a comunicaciones inalámbricas, servidores locales, servicios en la nube, entre otras propuestas.
\vspace{0.5cm}\\
Teniendo en cuenta lo anterior, muchos sistemas basados en diseños embebidos y/o tarjetas de desarrollo CPU se han propuesto como forma de integración y aplicación de sistemas de domótica en el hogar. La mayoría de estos proyectos o aplicativos son pensados para automatizar tareas específicas o agregarle una componente de comunicación remota al hogar. Si bien estos son puntos atractivos para un consumidor, el concepto de vivienda inteligente se puede expandir por más terrenos, tan diversos como el que se presentará en este documento.
\vspace{0.5cm}\\
Un concepto tan abierto y ventajoso como el de la vivienda inteligente o el internet de las cosas puede ser utilizado en el marco del concurso Solar Decatlón Latín América, el cual plantea el desarrollo de un proyecto de vivienda residencial amigable con el medio ambiente, donde el diseño arquitectónico ganador se decide en base a 10 indicadores cada uno con 100 puntos calificados de manera cualitativa y cuantitativa por los jueces.
\vspace{0.5cm}\\
Para la porción de la puntuación que es calificada de manera cuantitativa los jueces utilizan medidores que, dependiendo del indicador, puede corresponder para monitorear o medir el consumo eléctrico, temperatura, humedad, picos de consumo (3 kilo watts) donde se establece un valor de mínimo y máximo para un rango de puntos desde cero hasta la máxima calificación.
\vspace{0.5cm}\\
Considerando lo anterior, se plantea realizar la mejor aproximación tecnológica para cumplir de manera más acercada los indicadores de: eficiencia energética, innovación, balance eléctrico energético y condiciones de confort, puesto que son aquellos que pueden ser solucionados gestionados y monitoreados por tecnologías actuales. Basándose en lo expuesto anteriormente se plantea la siguiente pregunta al problema de ingeniería:
\vspace{0.5cm}\\
¿Cómo diseñar e implementar un sistema software que permita la gestión de cargas eléctricas y el monitoreo de variables físicas de manera local y remota para el concurso del Solar Decatlón latín América?.