\section{Marco Referencial}

En el marco de desarrollo de una propuesta de domótica y gestión inteligente es necesario apoyar los conceptos y las definiciones más significativas para el entendimiento del proyecto, como se presentan a continuación:\cite{Nosratabadi2017a}.

\subsection{Arquitectura.}
Se define como un conjunto de componentes de software que operan en conjunto utilizan interfaces entre ellos. Los componentes software utilizan un conjunto de herramientas de programación como APIs, Bibliotecas, Frameworks, para su óptimo diseño. Como aclaración las herramientas mencionadas anteriormente se definen como:

\begin{itemize}
	\item \textbf{API (Application Programming Interface):} Es un conjunto de subrutinas, funciones y procedimientos (o métodos, en la programación orientada a objetos) que ofrece cierta biblioteca para ser utilizado por otro software como una capa de abstracción.
    
	\item \textbf{Framework:} En el marco de desarrollo de software, un Framework (entorno de trabajo) corresponde a la estructura  modular que sirve de base para la organización y desarrollo de software. Normalmente, pueden incluir soporte para lenguajes de programación, librerías, entre otras. 
	\item \textbf{Biblioteca:} En programación, una biblioteca o librería es un conjunto de implementaciones funcionales, diseñadas en un lenguaje de programación específico, la cual ofrece una interfaz para la funcionalidad que se invoca.
\end{itemize}

Dentro de las arquitecturas que se diseñarán en el proyecto se utilizaran conceptos adicionales para referirse a algunas en específico; a continuación se mencionan las más importantes: \cite{Palizban2014}. 

\subsection{Servidor.}

Un módulo hardware y software con amplia capacidad de almacenamiento y procesamiento. Desde el punto de vista del software, los servidores son programas de computador que atienden las peticiones de otros programas llamados clientes. Dentro de estos llamados los principales servicios de un servidor son: compartir datos, información y recursos de hardware. Desde el punto de vista de hardware toda la información entrante de los dispositivos (clientes) es recibida a través de una interfaz de red y reconocida para su posterior procesamiento y almacenamiento.cite{Gaona2015}

\subsection{Aplicación web.}

Una aplicación web es un programa que se codifica en un lenguaje interpretable por los navegadores web, esto le permite a la aplicacion ser independiente del sistema operativo y depender de interpretador web (navegador). Dentro de los lenguajes utilizados como desarrollo para aplicaciones web se tienen: html, javascript, php, asp, python, ruby, etc.\cite{Gaona2015}.

\subsection{Aplicación de escritorio.} 

Una aplicacion escritorio es cualquier software que pueda ser instalado en un computador o sistema de cómputo, y que permita ejecutar ciertas rutinas. En este sentido una aplicación web puede ser una aplicación de escritorio. Los lenguajes de programación de aplicaciones de escritorio son un conjunto más amplio, los que más predominan son aquellos basados en máquinas virtuales o entornos de ejecución puesto que permiten realizar desarrollos independientes del sistema operativo. Dentro de los más conocidos están los lenguajes: Java, Python, C++, Golang, etc.
\vspace{0.5cm}\\
Dentro de los elementos materiales del sistema se encuentran los dispositivos que van a encargarse de la capa física durante el proceso de comunicación y adquisición de datos, a continuación se definirán los componentes más importantes de estos componentes.

\subsection{Dispositivo.} Teoria de colores
Un dispositivo es objeto hardware que puede ser visto como un sistema embebido con capacidades de procesamiento y comunicación a diferentes redes. En el dispositivo hardware reside el software que sirve como el manejador de las magnitudes físicas, voltajes, o sensores que están conectadas a él. Este componente normalmente posee las herramientas para el envío de información a servidores locales o remotos, en adición, dentro de los componentes que integran un dispositivo se tienen los siguientes:

\begin{itemize}
	\item \textbf{Sistema de archivos:} Este componente de software es el encargado de almacenar los datos del sistema ya sea localmente, en un servidor de manera remota. Además de lo anterior, este módulo sirve para el almacenamiento de la información necesaria para el funcionamiento del sistema operativo. 
    
	\item \textbf{Colector Cliente:} Este componente es el encargado de gestionar toda la información de los sensores que será enviada a manera de flujos de datos utilizando un protocolo de comunicación, esta capa de software añade los campos de metadata (control y gestión) para la efectiva comunicación.
	\item \textbf{Capa de Sensores:} Este software es el encargado de proveer todos los drivers lógicos para la coneccion de los elementos físicos conectados al dispositivo.
\end{itemize}

Debido al tiempo estipulado para el proyecto el dispositivo y los métodos de adquisición no serán de alta importancia en el desarrollo de la arquitectura y se asumirá que todo desarrollo de los dispositivos debe tener solucionada la comunicación con el servidor.
\vspace{0.5cm}\\
Desde el punto de vista del desarrollo de diseño gráfico es necesario conocer los conceptos que facilitan la implementación de una apropiada interfaz HMI. Dentro del marco del diseño, los conceptos de mayor importancia para este proyecto son:

\subsection{Teoria de colores.} 
La teoría del color es un grupo de reglas básicas en la mezcla de colores para conseguir el efecto emocional deseado combinando colores de luz o pigmentos. Dentro de esta teoría existen varios modelos que explican este proceso: el modelo RGB, CYMK, YIQ, HSI etc.
\\
Con el objetivo de comprender como se relacionan las emociones y los colores, se explicaran algunos términos comúnmente utilizados cuando se habla de colores y resultan necesarios para comprender el proceso de selección de la paleta de colores:

\begin{itemize}
	\item \textbf{Armonias de color:} Los colores armónicos son aquellos que funcionan bien juntos, es decir, que producen un esquema de color sensible al mismo sentido (la armonía nace de la percepción de los sentidos y, a la vez, esta armonía retro alimenta al sentido, haciéndolo lograr el máximo equilibrio que es hacer sentir tensión o relajación). 
	
	\item \textbf{Colores fríos:} En diseño, los colores fríos suelen usarse para dar sensación de tranquilidad, calma, seriedad y profesionalidad, también provocan la sensación de serenidad, recogimiento, la pasividad, el sentimentalismo, la sensación de frío. Como norma general son los colores que tienen azul y/o verde.
	
	\item \textbf{Colores cálidos} En diseño, los colores fríos suelen usarse para dar sensación de tranquilidad, calma, seriedad y profesionalidad. Como norma general, los colores cálidos son todos aquellos que van del rojo al amarillo, pasando por naranjas, marrones y dorados. 
	
	\item \textbf{Colores complementarios:}	Los colores complementarios son aquellos que se encuentran exactamente en el lugar opuesto del círculo cromático del modelo HSB. Es decir que cualquier color tiene su complemento a 180 grados de su valor HUE.
	
\end{itemize}





