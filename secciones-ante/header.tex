% header
\documentclass[12pt, titlepage,oneside]{article}
% ******** vmargin settings *********
\usepackage{vmargin} %This give you full control over the used page area, it maybe not the idea od Latex to do so, but I wanted to reduce to amount of white space on the page
%\setcitestyle{round,authoryear,citesep={;}, aysep={,}}
\setpapersize{USletter}
\setmarginsrb{4cm}%			%linker Rand, left edge --- izquierda
{2cm}%     %oberer Rand, top edge  --- arriba
{3cm}%		% -- derecha
{2cm}%   % -- abajo
{12pt}%			%Kopfzeilenhöhe, header hight
{1cm}%   	  %Kopfzeilenabstand, header distance
{12pt}%				%Fußzeilenhoehe footer hight
{1cm}%    	  %Fusszeilenabstand, footer distance   
% ********* Lenguage definition *******

\usepackage[spanish]{layout}
\usepackage[spanish,activeacute]{babel}
\usepackage[utf8]{inputenc}
\usepackage{hyperref}
\usepackage{enumerate}
\usepackage{amsmath}
\usepackage{amsfonts}
\usepackage{amssymb}
\usepackage{fancyhdr}
\usepackage{bbm}
\usepackage{tocloft}
\usepackage{array}
\usepackage{titlesec}
\usepackage{booktabs}
\usepackage[flushleft]{threeparttable}

\usepackage{makeidx}
\usepackage{lmodern}
\usepackage{kpfonts}
\usepackage{multirow}
\usepackage{booktabs}
% ********* Graphics definition *******
%\usepackage [dvips]             {graphicx}
%\usepackage{colortbl}
\usepackage[pdftex]{graphicx} % required to import graphic files
\usepackage{color} 	  % allows to mark some entries in the tables with color
\usepackage{eso-pic}  % these two are required to add the little picture on top of every page
\usepackage{everyshi} % these two are required to add the little picture on top of every page
\usepackage{apacite}
%\usepackage[left=2cm,right=2cm,top=2cm,bottom=2cm]{geometry}
\usepackage[numbers]{natbib}
\bibliographystyle{IEEEtranN}

% ******************* Entornos adicionales. ********************************
\setcounter{secnumdepth}{4}
\setcounter{tocdepth}{4}
\titleformat{\paragraph}
{\normalfont\normalsize\bfseries}{\theparagraph}{1em}{}
\titlespacing*{\paragraph}
{0pt}{3.25ex plus 1ex minus .2ex}{1.5ex plus .2ex}

\newtheorem{teo}{Teorema}
\newtheorem{defi}{Definici\'on}
\newtheorem{rem}{Remarque}
\newtheorem{col}{Coloquial}
\newtheorem{notacion}{Notaci\'on\'}
\newtheorem{eje}{Ejemplo}
\newtheorem{hip}{Hip\'otesis}
\newcommand{\bh}{\begin{hip}}
\newcommand{\eh}{\end{hip}}

\renewcommand{\contentsname}{Tabla de Contenido}
\renewcommand{\listtablename}{Lista de Tablas}
\renewcommand{\listfigurename}{Lista de Figuras}
\renewcommand{\refname}{REFERENCIAS}
\renewcommand{\tablename}{Tabla}
\renewcommand{\figurename}{Figura}
\renewcommand{\cfttabfont}{Tabla }
\renewcommand{\cftfigfont}{Figura }

\newcolumntype{C}[1]{>{\centering\let\newline\\\arraybackslash\hspace{0pt}}m{#1}} %Para centrar y ajustar tamaño de la columna de una tabla
\newcolumntype{L}[1]{>{\raggedright\let\newline\\\arraybackslash\hspace{0pt}}m{#1}} %Para alinear a la derecha y ajustar tamaño de la columna de una tabla
\newcolumntype{R}[1]{>{\raggedleft\let\newline\\\arraybackslash\hspace{0pt}}m{#1}}%Para alinear a la izquierda y ajustar tamaño de la columna de una tabla
\newcolumntype{M}{>{$\vcenter\bgroup\hbox\bgroup}c<{\egroup\egroup$}} %Para centrar verticalmente el contenido de una celda de una tabla
\typeout{Copyright Alejandro Gil Caicedo} % No cambiar esta linea-

\hyphenation{po-si-bles
al-can-za-bles al-can-za-bi-li-dad lo-cal-ment-te a-que-llas
re-pre-sen-ta-cio-nes cons-tan-te ins-tan-te mo-de-lo es-ta-dos
con-tro-la-ble con-tro-la-bi-li-dad de-sa-rro-llar-se re-fe-ren-cias
Co-lo-ra-do co-rri-en-te re-fe-ren-cia}

\fancyhf{}
\renewcommand{\headrulewidth}{0.5pt}
\renewcommand{\footrulewidth}{0.5pt}
%\setlength{\topmargin}{2.5 cm}
%\setlength{\headheight}{0.7 cm}
%\setlength{\headsep}{1 cm}
%\setlength{\textheight}{19 cm}
%\setlength{\oddsidemargin}{0.5 cm}
%\setlength{\footskip}{0.85 cm}
%\setlength{\textwidth}{16 cm}
%\fancyhead[LE,RO]{\bfseries\thepage}
\fancyhead[L]{\footnotesize\bfseries Implementacion de un manejador de recursos con inteligencia ambiental para una ...}
%\fancyhead[R]{\bfseries\leftmark}
%\frontmatter
%\fancyhead[R]{\footnotesize \bfseries Doctorado en Ingeniería, Pag.\thepage}
%\fancyhead[R]{\footnotesize\bfseries Énfasis en Automática, Pag.\thepage}
%\fancyhead[R]{\footnotesize\bfseries Énfasis en Eléctrica, Pag.\thepage}
%\fancyhead[R]{\footnotesize\bfseries Énfasis en Electrónica, Pag.\thepage}
\fancyhead[R]{\footnotesize\bfseries Pag. \thepage}
%\fancyfoot[L]{\footnotesize\textbf{Nombre del Autor}}

\fancyfoot[R]{\footnotesize\textbf{Propuesta de Trabajo de Grado}}
%\fancyfoot[R]{\footnotesize\textbf{Propuesta de Tesis - Doctorado-.}}

