\section{Antecedentes}

Alrededor de todo el mundo se han desarrollado sistemas de domótica basados en placas de desarrollo embebidas como Arduino, Raspberry, Intel, Amd, entre otros, estos desarrollos van desde los modelos más complejos y robustos hasta los económicos y sencillos. En este apartado se expondrán y compararán las propuestas existentes en la literatura con el proyecto planteado para este proyecto.

\subsection{En America Latina.}
Para el contexto Colombiano el documento “Integración de los sistemas embebidos Raspberry Pi y Arduino para el manejo de un brazo robótico mediante una aplicación Android” expuesto por David R. Suares y otros \cite{Rolando2014}, Explica cómo se pueden comunicar  los sistemas embebidos de desarrollo; Arduino, Raspberry, Android, y un brazo robótico, a través de algoritmos programados en Python y C. En el artículo diseñan una interfaz gráfica en Android utilizando el lenguaje de programación Java como solución HMI, esta posee una rudimentaria visión de los sensores y actuadores del sistema, sin teoría de colores o formas.
\vspace{0.5cm}\\
Por otra parte, en México, J. E. G. Salas y otros \cite{GiralSala2016} desarrollaron el proyecto; “Smart switch to connect and disconnect electrical devices at home”, donde se implementó un sistema embebido usando la gama de controladores Econias para la realización de un módulo de domótica totalmente independiente de cualquier sistema centralizado de control. Las características del proyecto lo definen como un  Smart Switch (SS), capaz de conectar o desconectar vía remota, haciendo uso de internet, cualquier aparato eléctrico que se pueda energizar a través del toma corriente del hogar.
\vspace{0.5cm}\\
Además del anterior, para la ponencia “International Autumn Meeting on Power, Electronics and Computing” también en México, los ingenieros Julio Cabrera, María Mena y otros \cite{Cabrera2017} diseñaron un sistema basado en Arduino llamado “Intelligent Assistant to Control Home Power Network” con el que, a través de una aplicación web se realiza el control de los circuitos del hogar y la adquisición de algunas señales análogas; el sistema se compone de una cámara y un micrófono para la seguridad del hogar
\vspace{0.5cm}\\

\subsection{En el mundo.}
En lo que concierne al contexto internacional se observa cómo se mantiene una predominancia en el uso de cpu’s de bajo costo como la Raspberry pi, o embebidos de arduino como el Arduino Leonardo. Aunque para aplicaciones de ioT se puede encontrar  que en el mercado existen tecnologías de Texas Instruments, Lantronix, BlueGiga, Microchip, Econais, Murata Electronics, Raspberry, entre otras.
\vspace{0.5cm}\\
Por ejemplo, en el artículo publicado por Ahmed Imteaj, Tanveer Rahman y otros titulado “An IoT based fire alarming and authentication system for workhouse using Raspberry Pi 3” \cite{Imteaj2017} se muestra una aplicación directa del uso de una Raspberry integrada con varios módulos de Arduino. En el artículo exponen un sistema capaz de detectar incendios y la habilidad de enviar a través de una señal celular una imagen del incendio de manera automática. Dentro de las capacidades de los módulos está: la comunicación wifi entre los Arduinos y la raspberry, también un sistema de comunicación GSM para la interacción con el usuario final. Además cada módulo tiene sus correspondientes entradas análogas para los sensores de carbono y luz y salidas para el encendido de una alarma.
\vspace{0.5cm}\\
Otra aplicación se puede observar en el artículo “Based Urban Climate Monitoring using Raspberry Pi” \cite{Shete2016} en el cual se muestra como es utilizado el sistema embebido Raspberry Pi para el monitoreo remoto de variables ambientales utilizando un módulo de adquisición de datos implementado en arduino, y un modelo web para el monitoreo de datos de manera remota a través de un servidor local computado en la Raspberry.
\vspace{0.5cm}\\


