\subsection{Lineamientos de desarrollo}
Debido a la naturaleza del proyecto, parte de los aspectos definidos al momento de establecer el anteproyecto fueron unas necesidades funcionales explicadas en el formato \ref{tab0}.

% Please add the following required packages to your document preamble:
% \usepackage[table,xcdraw]{xcolor}
% If you use beamer only pass "xcolor=table" option, i.e. \documentclass[xcolor=table]{beamer}

\begin{longtable}[c]{lllc}
\hline
\multicolumn{1}{|l|}{
\begin{tabular}[c]{@{}l@{}}
	\includegraphics[width=1cm]{figuras/uv_logo.png}
\end{tabular}
} 
& 
\multicolumn{2}{l|}{\cellcolor[HTML]{808080}{\color[HTML]{FFFFFF} \textbf{
\begin{tabular}[c]{@{}l@{}}Universidad del Valle\\--Implementación de un gestionado de recursos\\ para el hogar con inteligencia ambiental. ---\end{tabular}}}}
&
\multicolumn{1}{c|}{\textbf{\begin{tabular}[c]{@{}c@{}}Rev:\\ 003\end{tabular}}} \\ \hline
\endfirsthead
%
\endhead
%
\multicolumn{2}{|l|}{\textbf{\begin{tabular}[c]{@{}l@{}}Título:\\   ESPECIFICACIÓN DE \\ REQUERIMIENTOS \\ FUNCIONALES\end{tabular}}} & \multicolumn{1}{l|}{\textbf{\begin{tabular}[c]{@{}l@{}}Documento :\\ ERF-000\end{tabular}}} & \multicolumn{1}{c|}{\textbf{\begin{tabular}[c]{@{}c@{}}Página :\\ 1 de 1\end{tabular}}} \\ \hline
 &  &  & \multicolumn{1}{l}{} \\ \hline
\rowcolor[HTML]{808080} 
\multicolumn{4}{|c|}{\cellcolor[HTML]{808080}\textbf{REVISIÓN HISTÓRICA}} \\ \hline
\rowcolor[HTML]{808080} 
\multicolumn{1}{|l|}{\cellcolor[HTML]{808080}\textbf{Rev.}} & \multicolumn{1}{l|}{\cellcolor[HTML]{808080}\textbf{Descripción del Cambio}} & \multicolumn{1}{l|}{\cellcolor[HTML]{808080}\textbf{Autor}} & \multicolumn{1}{l|}{\cellcolor[HTML]{808080}\textbf{Fecha}} \\ \hline
\multicolumn{1}{|l|}{001} & \multicolumn{1}{l|}{Construcción del documento} & \multicolumn{1}{l|}{\begin{tabular}[c]{@{}l@{}}Juan David Ramirez\\ Villegas\end{tabular}} & \multicolumn{1}{l|}{18/03/2017} \\ \hline
\multicolumn{1}{|l|}{002} & \multicolumn{1}{l|}{Correcciones} & \multicolumn{1}{l|}{\begin{tabular}[c]{@{}l@{}}Juan David Ramírez\\ Villegas\end{tabular}} & \multicolumn{1}{l|}{15/11/2017} \\ \hline
\multicolumn{1}{|l|}{003} & \multicolumn{1}{l|}{Revisión} & \multicolumn{1}{l|}{\begin{tabular}[c]{@{}l@{}}Juan David Ramírez\\ Villegas\end{tabular}} & \multicolumn{1}{l|}{18/12/2017} \\ \hline
 &  &  & \multicolumn{1}{l}{} \\ \hline
\rowcolor[HTML]{808080} 
\multicolumn{1}{|l|}{\cellcolor[HTML]{808080}\textbf{Ref \#}} & \multicolumn{2}{l|}{\cellcolor[HTML]{808080}\textbf{Funciones}} & \multicolumn{1}{l|}{\cellcolor[HTML]{808080}\textbf{Categoría}} \\ \hline
\multicolumn{1}{|l|}{1.0} & \multicolumn{2}{l|}{USER APP (application móvil)} & \multicolumn{1}{c|}{\textbf{O}} \\ \hline
\multicolumn{1}{|l|}{1.1} & \multicolumn{2}{l|}{\begin{tabular}[c]{@{}l@{}}El usuario debe poder acceder a la \\ información medida en tiempo real.\end{tabular}} & \multicolumn{1}{c|}{\textbf{E}} \\ \hline
\multicolumn{1}{|l|}{1.2} & \multicolumn{2}{l|}{\begin{tabular}[c]{@{}l@{}}El usuario debe poder acceder a los datos \\ históricos recopilados en el mes y la \\ relación de sus gastos con ellos.\end{tabular}} & \multicolumn{1}{c|}{\textbf{O}} \\ \hline
\multicolumn{1}{|l|}{1.3} & \multicolumn{2}{l|}{\begin{tabular}[c]{@{}l@{}}El usuario debe poder cambiar los \\ parámetros de configuración escogidos\\  por defecto para la vivienda del\\  Solar Decathlon.\end{tabular}} & \multicolumn{1}{c|}{\textbf{E}} \\ \hline
\multicolumn{1}{|l|}{1.4} & \multicolumn{2}{l|}{\begin{tabular}[c]{@{}l@{}}El usuario debe recibir recomendaciones\\  para mejorar su entorno y su huella \\ hídrica cada cierto tiempo.\end{tabular}} & \multicolumn{1}{c|}{\textbf{E}} \\ \hline
\multicolumn{1}{|l|}{1.5} & \multicolumn{2}{l|}{\begin{tabular}[c]{@{}l@{}}La aplicación notificará al usuario\\ cuando la factura de energía o agua\\  esté vencida.\end{tabular}} & \multicolumn{1}{c|}{\textbf{O}} \\ \hline
\multicolumn{1}{|l|}{1.6} & \multicolumn{2}{l|}{\begin{tabular}[c]{@{}l@{}}El usuario debe ser capaz de ver un \\ índice del impacto ambiental de su\\  estilo de vida basado en datos \\ cualitativos y cuantitativos\end{tabular}} & \multicolumn{1}{c|}{\textbf{E}} \\ \hline
\multicolumn{1}{|l|}{1.7} & \multicolumn{2}{l|}{\begin{tabular}[c]{@{}l@{}}El sistema debe ser capaz de notificar \\ las pérdidas eléctricas, y por consiguiente\\  económicas, de un mal uso de los horarios\\  establecidos por defecto.\end{tabular}} & \multicolumn{1}{c|}{\textbf{O}} \\ \hline
\multicolumn{1}{|l|}{2.0} & \multicolumn{2}{l|}{HOUSE MANAGER (Aplicación en sitio)} & \multicolumn{1}{c|}{\textbf{E}} \\ \hline
\multicolumn{1}{|l|}{2.1} & \multicolumn{2}{l|}{\begin{tabular}[c]{@{}l@{}}El sistema debe mostrar gráficamente \\ la cantidad de agua y potencia consumida\\  durante el día contrastándolas con un \\ valor máximo recomendado.\end{tabular}} & \multicolumn{1}{c|}{\textbf{E}} \\ \hline
\multicolumn{1}{|l|}{2.2} & \multicolumn{2}{l|}{\begin{tabular}[c]{@{}l@{}}Por defecto, el uso de las cargas \\ eléctricas más significativas debe\\  programarse durante los picos de \\ generación en la casa. Estos horarios \\ podrán ser modificados bajo una \\ advertencia de uso no eficiente.\end{tabular}} & \multicolumn{1}{c|}{\textbf{E}} \\ \hline
\multicolumn{1}{|l|}{2.3} & \multicolumn{2}{l|}{\begin{tabular}[c]{@{}l@{}}El usuario debe poder acceder \\ a la información medida en tiempo real.\end{tabular}} & \multicolumn{1}{c|}{\textbf{E}} \\ \hline
\multicolumn{1}{|l|}{2.4} & \multicolumn{2}{l|}{\begin{tabular}[c]{@{}l@{}}El sistema debe poder comunicarse \\ con una base de datos que represente \\ todas las variables y el estado de los \\ circuitos de la vivienda.\end{tabular}} & \multicolumn{1}{c|}{\textbf{E}} \\ \hline
\caption{Requerimientos Funcionales del sistema}
\label{tab0}\\
\end{longtable}
La columna de la categoría indica si el requerimiento indicado es uno esencial (E) u opcional (O) para el desarrollo del sistema. Este documento será de gran importancia puesto que definió los lineamientos de desarrollo de todas las aplicaciones.


