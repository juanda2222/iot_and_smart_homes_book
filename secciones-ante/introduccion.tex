\subsection{Contexto introductorio}
No es para nadie una sorpresa que durante este siglo, las nuevas generaciones que nacen rodeadas de la tecnología, están cambiando las costumbres y el paradigma actual de lo que significa una vivienda, con una sociedad cada vez más relacionada con la tecnología, nuevos desafíos de diseño e implementación se plantean, en el sentido de cumplir las expectativas de estas nuevas generaciones.
\vspace{0.5cm}\\
Básicamente el más importante beneficio utilizar la tecnología en viviendas comerciales es el proveer de servicios y facilidades a personas discapacitadas y ancianos [3], por otra parte, tenemos beneficios de monitorización y la capacidad de añadir herramientas que permitan la integración con redes eléctricas inteligentes. La popularidad de sistemas inteligentes ha venido incrementando debido al confort adicional, y herramientas de seguridad que pueden recibir los usuarios. 
\vspace{0.5cm}\\
Por lo tanto, hay ciertos factores que deben ser tenidos en cuenta a la hora de diseñar un sistema de control y monitoreo de la vivienda, En este documento se explicara el proceso de diseño e implementacion de un software para el manejo inteligente de una vivienda del Solar Decathlon Latinoamerica utilizando diferentes tecnologias de frontend y backend
\vspace{0.5cm}\\
La arquitectura desarrollada se puede describir en 3 productos: 

\begin{enumerate}
	\item Un servicio de backend serverles basado en, un broker de mqtt, un servicio de almacenamiento de datos tipo S3, una base de datos relacional tipo mysql e interfaces http basadas en cloud functions. 
	
	\item Un programa encargado de administrar y controlar todas las rutinas de comunicacion, reporte y almacenamiento en el sitio de control con su respectiva interfaz de usuario.
	
	\item Una aplicacion celular capaz de monitorear y controlar los datos generados localmente y los actuadores de la vivienda del Solar Decathlon.
\end{enumerate}
	

