	
\customsection{Conclusiones y Trabajos futuros}	

\subsection{Conclusiones}
\begin{itemize}
	
	\item Cuando Se realizan aplicaciones de escritorio, como la desarrollada en este documento, con tecnologías como Java o Python existen pocas herramientas para mejorar la experiencia gráfica de la interfaz. Por el contrario en un entorno web la comunidad entorno a mejorar el sistema de interfaces es excepcional. Teniendo en cuenta lo anterior, desarrollar una aplicación de escritorio basada en alguno de estos dos lenguajes es viable si la aplicación es pequeña, sin embargo, si se desea escalar el aplicativo, es más apropiado utilizar un frameworks web con un backend local.
	
	\item Dentro de las posibles maneras de implementar productos, que necesiten algún tipo de arquitectura web como backend, el diseño basado en los servicios serverless de algún proveedor como; Amazon, Google o Azure, es la mejor alternativa si la empresa o equipo de desarrollo no posee capital económico o logístico para mantener un servidor de manera local. Es decir, los sistemas severless representan una gran alternativa al modelo tradicional si se trata de un equipo de desarrollo pequeño. Por ejemplo, un equipo en proceso de emprendimiento o una pymes sin capital para mantener todo el modelo clásico del servidor.
	
	\item Debido a que React Native ofrece la posibilidad de desarrollar aplicaciones de múltiple plataforma con interfaces graficas iguales usando el mismo código, los costos de implementación son más bajos que tener dos equipos por separado, uno para Android y otro para IOS. Por ende, si una empresa cuenta con el presupuesto, puede manejar dos equipos, de lo contrario resulta más eficiente capacitar un equipo en React Native.
	
	\item Teniendo en cuenta que React Native es una tecnología muy moderna, todavía no cuenta con una comunidad lo suficientemente grande como para garantizar que todas las herramientas de programación estén implementadas y tengan un equipo formal que las mantenga. Debido a lo anterior, al momento de escoger React Native como entorno de desarrollo, es muy importante identificar todas las herramientas necesarias para implementar la aplicación; si todas las herramientas tienen un buen equipo de soporte en la web, React Native es el camino más apropiado para su implementación.
	
	\item Una vivienda del futuro debe tener la capacidad no solo de controlarse y monitorearse sino que también debe considerar su impacto ambiental, es decir, las casas del futuro deben transmitir la importancia de adquirir costumbres más amigables con el medio ambiente.
	
\end{itemize}

\subsection{Trabajos futuros.}

Los servicios orientados a ``Iot'' se encuentran en crecimiento y cada vez más industrias se suman a la automatización de procesos basados en servicios orientados a internet. Por ende, la demanda de sistemas y modelos de comunicación escalables y seguros se verá incrementada de manera sostenible, y modelos como el implementado en este documento jugarán un papel importante en la transformación de estas industrias.

\subsubsection {Pruebas con más dispositivos.}
Debido a los límites de este proyecto de grado, y las pruebas realizadas para comprobar su funcionamiento; el trabajo futuro más relevante para extender el alcance de este proyecto es conectar más dispositivos al sistema. El anterior proceso involucraría utilizar los dos clientes ya implementados (el aplicativo móvil y el de escritorio) para crear una red de usuarios que pueda usar todas las capacidades del sistema. Como aclaración, este proceso involucraría gastos asociados a la cantidad de almacenamiento utilizada y las cuotas de flujos de datos mínimas permitidas para que el sistema se mantenga gratis en el proveedor de servicios de nube utilizado.


\subsubsection {Integración de sensores y actuadores al sistema.}
Teniendo en cuenta los objetivos del proyecto y como están encaminados a la competencia “Solar Decathlon Latinoamerica”, un trabajo futuro es integrar los sensores y actuadores que le permitan al sistema aprovechar todas sus características en la vivienda del Solar Decathlon; es decir, implementar desarrollos de hardware que realicen la medición real de las variables importantes para la competencia como: consumo eléctrico, consumo de agua, temperatura y humedad y velocidad del viento.

\subsubsection  {Interacción con la interfaz de voz de Google usando ``Dialog flow''.}
Finalmente, para complementar el componente de inteligencia de la vivienda, se podría integrar una interfaz por voz que cumpla con las siguientes características: que sea fácil de usar para el usuario, que sea familiar a su contexto, que le permita al usuario hacer cambios en los actuadores del sistema, que le permita conocer el estado actual de los sensores y que los comandos de voz se puedan utilizar de manera natural.


