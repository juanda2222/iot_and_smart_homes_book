

\section*{\centering {IMPLEMENTACI\'ON DE UN MANEJADOR DE RECURSOS CON INTELIGENCIA AMBIENTAL PARA UNA VIVIENDA DEL SOLAR DECATHLON LATIN AMERICA 2019}}
\addcontentsline{toc}{section}{Resumen}

En este documento se explicará el proceso realizado para implementar un modelo cliente servidor, que comunique diferentes dispositivos bajo el paradigma de internet de las cosas. El sistema consta principalmente de 3 productos: el producto del backend, una aplicación móvil y una aplicación de escritorio. La aplicación móvil se encargó, a grandes rasgos, de monitorear y controlar de manera remota todos los dispositivos conectados al sistema. El aplicativo en sitio se encargó de gestionar y ordenar, inclusive con perdidas de conexion, las mediciones y los mensajes de control. Por último, el servicio del backend se encargó de comunicar ambos productos de frontend y almacenar la información de los usuarios.
\vspace{0.5cm}\\
Teniendo en cuenta lo anterior, la arquitectura se diseñó bajo la posibilidad de soportar hasta 4000 dispositivos entre los cuales se encuentran computadores, celulares y cualquier dispositivo capaz de ejecutar rutinas en Python o Nodejs. Ademas, ambos aplicativos clientes fueron diseñados para ser multiplataforma. Por el alcance del proyecto no se realizaron pruebas a gran escala, es decir, con más de 4 dispositivos conectados.
\begin{center}
	\textsl{Palabras clave:} serverless, Json web token, comunicaciones móviles, comunicación industrial, control industrial, gestión de redes.
\end{center}

\newpage

\section*{\centering {IMPLEMENTATION OF A RESOURCE MANAGER WITH ENVIRONMENTAL INTELLIGENCE FOR A SOLAR DECATHLON LATIN AMERICA 2019}}

This document will explain the process carried out to implement a client server model that communicates different devices under the internet of things paradigm. The system consists mainly of 3 products: the backend product, a mobile application and a desktop application. The mobile application was, in broad strokes, responsible for remotely monitoring and controlling all the devices connected to the system. The application on site was responsible for managing and ordering, even with a lost connection, the measurements and control messages. Finally, the backend service was responsible for communicating both frontend products and storing user information.
\vspace {0.5cm} \\
Taking into account the above, the architecture was designed under the possibility of supporting up to 4000 devices among which are computers, cell phones and any device capable of running routines in Python or Nodejs. In addition, both client applications were designed to be cross-platform. Due to the scope of the project, no large-scale tests were performed, that is, with more than 4 devices connected.

\begin{center}
\textsl{Keywords:} serverless, Json web token, mobile comunication, industrial comunication, industrial control, computer network management
\end{center}