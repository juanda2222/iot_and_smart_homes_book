\subsection{Resumen}
En este documento se explicara el proceso realizado para implementar una arquitectura sofware para la comunicacion de diferentes dispositivos bajo el paradigma de internet de las cosas. El sistema consta principalmente de 3 productos: el producto del backend, una aplicacion movil y una aplicación de escritorio. La aplicacin movil se encargo de monitorear y controlar todos los dispositivos conectados al sistema´. El aplicativo en sitio se encargo de manejar los sensores y actuadores de manera local en caso de perdida de conexion, de manera remota en presencia de una red wifi y otras funciones adicionales. Por ultimo, el servicio del backend se encargo de comunicar ambos productos de frontend y almacenar la informacion de los usuarios.
\vspace{0.5cm}\\
El producto del backend se implemento en el servicio de Google Cloud Services y utilizó los siguientes elementos para su funcionamiento: Un "Storage version 3 system" conectado a la red privada de Google para el almacenamiento de archivos, Una base de datos relacional MySql para el almacenamiento de datos de monitoreo y control de los dispositivos, Una lista de "cloud functions" activadas por solicitudes http y por ellas mismas que se comunican directamente con la base de datos y finalmente, un broker nativo de Google Cloud llamado "Cloud IOT" que recibe las conexiones de mqtt de los dispositivos del sistema.
\vspace{0.5cm}\\
La aplicación movil desarrollada constó de principalmente de 3 escenas: la esccena de control, utilizada para el manejo de los actuadores disponibles desde el aplicativo en sitio, la escena de medicion, encargada de presentarle al usuario las mediciones adquiridas por el aplicativo en sitio en tiempo real y la escena principal, donde se puede observar un indicador del porcentaje de las variables mas relevantes respecto a un punto de referencia. Por ejemplo, el porcentaje de consumo de agua con respecto al promedio de consumo de agua en en un mes de un grupo familiar.
\vspace{0.5cm}\\
Por otra parte, el aplicativo en sitio se desarrollo usando python y se diseño para manejar un flujo de datos mas grandes y la posibilidad de funcionar desconectado de la red. La aplicacion le permitio al usuario programar rutinas de control de los circuitos a partir de horarios, visualizar los datos medidos, cambiar parametros de adquisicion como el tamaño del bufer y la frecuencia de muestreo, entre otros.
\vspace{0.5cm}\\
Finalmente la arquitectura se diseño bajo la posibilidad de soportar hasta 4000 dispositivos de alto trafico de datos entre los cuales se encuentran computadores, celulares y cualquier dispositivo capaz de ejecutar rutinas en python o nodejs. 
