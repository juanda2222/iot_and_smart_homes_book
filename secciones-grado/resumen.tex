

\section*{\centering {IMPLEMENTACI\'ON DE UN MANEJADOR DE RECURSOS CON INTELIGENCIA AMBIENTAL PARA UNA VIVIENDA DEL SOLAR DECATHLON LATIN AMERICA 2019}}
\addcontentsline{toc}{section}{Resumen}

En este documento se explicará el proceso realizado para implementar una arquitectura software, que comunique diferentes dispositivos bajo el paradigma de internet de las cosas. El sistema consta principalmente de 3 productos: el producto del backend, una aplicación móvil y una aplicación de escritorio. La aplicación móvil se encargó, a grandes rasgos, de monitorear y controlar de manera remota todos los dispositivos conectados al sistema. El aplicativo en sitio se encargó principalmente de manejar los sensores y actuadores de manera local, en caso de pérdida de conexión, y de manera remota en presencia de una red wifi. Por último, el servicio del backend se encargó de comunicar ambos productos de frontend y almacenar la información de los usuarios.
\vspace{0.5cm}\\
Teniendo en cuenta lo anterior, la arquitectura se diseñó bajo la posibilidad de soportar hasta 4000 dispositivos entre los cuales se encuentran computadores, celulares y cualquier dispositivo capaz de ejecutar rutinas en Python o Nodejs. Pero por el alcance del proyecto no se realizaron pruebas asociadas a esta característica.
\begin{center}
	\textsl{Palabras clave:} serverless, Json web token, comunicaciones móviles, comunicación industrial, control industrial, gestión de redes.
\end{center}

\newpage

\section*{\centering {IMPLEMENTATION OF A RESOURCE MANAGER WITH ENVIRONMENTAL INTELLIGENCE FOR A SOLAR DECATHLON LATIN AMERICA 2019}}

This document will explain the process carried out to implement a software architecture that communicates different devices under the internet of things paradigm. The system consists mainly of 3 products: the backend product, a mobile application and a desktop application. The mobile application was responsible, in a general manner, for monitoring and controlling remotely all devices connected to the system. The desktop application was mainly responsible for handling the sensors and actuators locally, in the case of a lost connection, and remotely in the presence of a Wi-Fi network. Finally, the backend service was responsible for communicating both frontend products and storing user information.
\vspace{0.5cm}\\
Taking into account the above, the architecture was designed with the possibility of supporting up to 4000, including computers, cell phones and any device capable of executing routines in Python or Nodejs. But due to the scope of the project, no tests associated with this characteristic were performed.

\begin{center}
\textsl{Keywords:} serverless, Json web token, mobile comunication, industrial comunication, industrial control, computer network management
\end{center}