\section{introduccion}



\section*{\centering {IMPLEMENTACI\'ON DE UN MANEJADOR DE RECURSOS CON INTELIGENCIA AMBIENTAL PARA UNA VIVIENDA DEL SOLAR DECATHLON LATIN AMERICA 2019}}
\addcontentsline{toc}{section}{Resumen}

En este documento se explicará el proceso realizado para implementar un modelo cliente servidor, que comunique diferentes dispositivos bajo el paradigma de internet de las cosas. El sistema consta principalmente de 3 productos: el producto del backend, una aplicación móvil y una aplicación de escritorio. La aplicación móvil se encargó, a grandes rasgos, de monitorear y controlar de manera remota todos los dispositivos conectados al sistema. El aplicativo en sitio se encargó de gestionar y ordenar, inclusive con perdidas de conexion, las mediciones y los mensajes de control. Por último, el servicio del backend se encargó de comunicar ambos productos de frontend y almacenar la información de los usuarios.
\vspace{0.5cm}\\
Teniendo en cuenta lo anterior, la arquitectura se diseñó bajo la posibilidad de soportar hasta 4000 dispositivos entre los cuales se encuentran computadores, celulares y cualquier dispositivo capaz de ejecutar rutinas en Python o Nodejs. Ademas, ambos aplicativos clientes fueron diseñados para ser multiplataforma. Por el alcance del proyecto no se realizaron pruebas a gran escala, es decir, con más de 4 dispositivos conectados.
\begin{center}
	\textsl{Palabras clave:} serverless, Json web token, comunicaciones móviles, comunicación industrial, control industrial, gestión de redes.
\end{center}

\newpage

\section*{\centering {IMPLEMENTATION OF A RESOURCE MANAGER WITH ENVIRONMENTAL INTELLIGENCE FOR A SOLAR DECATHLON LATIN AMERICA 2019}}

This document will explain the process carried out to implement a client server model that communicates different devices under the internet of things paradigm. The system consists mainly of 3 products: the backend product, a mobile application and a desktop application. The mobile application was, in broad strokes, responsible for remotely monitoring and controlling all the devices connected to the system. The application on site was responsible for managing and ordering, even with a lost connection, the measurements and control messages. Finally, the backend service was responsible for communicating both frontend products and storing user information.
\vspace {0.5cm} \\
Taking into account the above, the architecture was designed under the possibility of supporting up to 4000 devices among which are computers, cell phones and any device capable of running routines in Python or Nodejs. In addition, both client applications were designed to be cross-platform. Due to the scope of the project, no large-scale tests were performed, that is, with more than 4 devices connected.

\begin{center}
\textsl{Keywords:} serverless, Json web token, mobile comunication, industrial comunication, industrial control, computer network management
\end{center}
\subsection{Contexto introductorio}
Durante este siglo, las nuevas generaciones que nacen rodeadas de la tecnología están cambiando las costumbres y el paradigma actual de lo que significa una vivienda. Con una sociedad cada vez más relacionada con la tecnología, nuevos desafíos de diseño e implementación se plantean para satisfacer las expectativas tecnologicas de las nuevas generaciones.
\vspace{0.5cm}\\
Básicamente el más importante beneficio utilizar la tecnología en viviendas comerciales es el proveer de servicios y facilidades a personas discapacitadas y ancianos \cite{GiralSala2016}, por otra parte, la adición de herramientas meidición y control permite ha permitido crear modelos de ciudad inteligente e interconectada. La popularidad de sistemas inteligentes ha venido incrementando debido al confort adicional, y herramientas de seguridad que pueden recibir los usuarios. 
\vspace{0.5cm}\\
Por lo tanto, hay ciertos factores que deben ser tenidos en cuenta a la hora de diseñar un sistema de control y monitoreo de la vivienda, en este documento se explicará el proceso de diseño e implementación de un software para el manejo inteligente de una vivienda del Solar Decathlon Latinoamérica utilizando diferentes tecnologías de \textit{frontend} y \textit{backend}
\vspace{0.5cm}\\
La arquitectura desarrollada se puede describir en 3 productos:
 
\begin{enumerate}
	\item Un servicio de \textit{backend} serverless basado en: un broker de Mqtt, un servicio de almacenamiento de datos tipo S3, una base de datos relacional tipo MySql e interfaces http basadas en funciones de nube.
	\item Un programa encargado de administrar y controlar todas las rutinas de comunicación, reporte y almacenamiento en el sitio de control con su respectiva interfaz de usuario.
	\item Una aplicación celular capaz de monitorear y controlar los datos generados localmente y los actuadores de la vivienda del Solar Decathlon.
\end{enumerate}


\subsection{Planteamiento del problema}

Hoy en día las tecnologías orientadas internet y la mayoría de los objetos utilizados en la cotidianidad, en general, están cada vez más relacionadas entre sí; el pasado año se vio un incremento de 2.79 billones de nuevos dispositivos conectados a la red de redes (entre los cuales no solo se encuentran celulares computadores y tablets). Nuevos dispositivos como aires acondicionados, televisores y electrodomésticos en general son lanzados al mercado con la capacidad de comunicarse con el usuario a través de aplicaciones web o dispositivos móviles día tras día. Lo anterior se debe al aumento en los desarrollos de hardware y software para sistemas con la capacidad de acceso a la red; es decir, con la capacidad de acceder a comunicaciones inalámbricas, servidores locales, servicios en la nube, entre otras propuestas.
\vspace{0.5cm}\\
Teniendo en cuenta lo anterior, muchos sistemas basados en diseños embebidos y/o tarjetas de desarrollo CPU se han propuesto como forma de integración y aplicación de sistemas de domótica en el hogar. La mayoría de estos proyectos o aplicativos son pensados para automatizar tareas específicas o agregarle una componente de comunicación remota al hogar. Si bien estos son puntos atractivos para un consumidor, el concepto de vivienda inteligente se puede expandir por más terrenos, tan diversos como el que se presentará en este documento.
\vspace{0.5cm}\\
Un concepto tan abierto y ventajoso como el de la vivienda inteligente o el internet de las cosas puede ser utilizado en el marco del concurso Solar Decatlón Latín América, el cual plantea el desarrollo de un proyecto de vivienda residencial amigable con el medio ambiente, donde el diseño arquitectónico ganador se decide en base a 10 indicadores cada uno con 100 puntos calificados de manera cualitativa y cuantitativa por los jueces.
\vspace{0.5cm}\\
Para la porción de la puntuación que es calificada de manera cuantitativa los jueces utilizan medidores que, dependiendo del indicador, puede corresponder para monitorear o medir el consumo eléctrico, temperatura, humedad, picos de consumo (3 kilo watts) donde se establece un valor de mínimo y máximo para un rango de puntos desde cero hasta la máxima calificación.
\vspace{0.5cm}\\
Considerando lo anterior, se plantea realizar la mejor aproximación tecnológica para cumplir de manera más acercada los indicadores de: eficiencia energética, innovación, balance eléctrico energético y condiciones de confort, puesto que son aquellos que pueden ser solucionados gestionados y monitoreados por tecnologías actuales. Basándose en lo expuesto anteriormente se plantea la siguiente pregunta al problema de ingeniería:
\vspace{0.5cm}\\
¿Cómo diseñar e implementar un sistema software que permita la gestión de cargas eléctricas y el monitoreo de variables físicas de manera local y remota para el concurso del Solar Decatlón latín América?.
\subsection{Justificación}
Respecto a la vivienda inteligente o la domótica en el hogar, se puede decir que la literatura es amplia y que existen diferentes propuestas comerciales de desarrollo como lo son: Calaos, Domoticz, Home Assistant, OpenHAB, OpenMotics, donde cada marca propone su marco de trabajo, y diferentes plataformas de desarrollo tanto software como hardware. Una desventaja de estas propuestas es su interfaz hardware, en la mayoría de los casos el hardware adicional corresponde a dispositivos desarrollados y mantenidos por ellos mismos, lo que significa que limita la escalabilidad de las
\vspace{0.5cm}\\
Las anteriores tecnologías son propuestas donde un centro de mando controla dispositivos externos, pero, por otra parte, si pensamos en cada módulo hardware con la capacidad de conectarse a la red de redes de manera independiente (Iot) como lo hace la propuesta realizada el en proyecto del “Smart switch” \cite{GiralSala2016}; se tiene como ventaja la independencia sobre las características físicas del hogar o de la capacitación técnica del instalador del sistema, pero como desventaja, al momento de escalar el problema a un mayor número de circuitos se pueden presentar problemas de saturación de la red inalámbrica (wifi) y sería más costoso debido al hardware adicional para cada nuevo circuito que se desee controlar.
\vspace{0.5cm}\\
Teniendo en cuenta las comparaciones anteriores, se podría pensar que un sistema que englobe varios sub sistemas podría ser el adecuado desde el punto de vista de la escalabilidad, pero esa no es la única razón; Hoy en día los métodos de generación alternativa y energía renovable se encuentran en crecimiento, por lo que nace la necesidad de interconectar redes eléctricas inteligentes, es decir, una red con un componente de generación y consumo tanto como AC y DC. Este tipo de topología eléctrica se le conoce como nanogrid, y se ampliará la teoría respecto a este concepto en futuras secciones del documento. Esta topología eléctrica añade como variable la generación eléctrica junto con una nueva forma fisica de energía (corriente directa).
\vspace{0.5cm}\\
Inclusive, se puede observar que artículos académicos se han dedicado a analizar este concepto en el contexto Colombiano,  por ejemplo un estudio sobre microgrids mencionó la importancia y necesidad de implementar viviendas inteligentes como se ve a continuación: “desde el contexto de seguridad eléctrica, equidad social y mitigación del impacto ambiental en Colombia, el sistema energético debe afrontar los nuevos retos requeridos para satisfacer la demanda. Desde un punto de vista técnico, es necesario dotar la red tradicional con las características de una red inteligente ”. 
\vspace{0.5cm}\\
Pero ¿acaso una red inteligente implica una casa inteligente? La respuesta es; parcialmente sí. Considerando que hoy en día toda vivienda realiza un monitoreo o medición de al menos 2 variables críticas; el consumo eléctrico, y el consumo de agua, ambos son de interés para la compañía prestadora de servicios públicos en la vivienda.
\vspace{0.5cm}\\
Resaltando la importancia de las anteriores variables físicas y teniendo en cuenta el modelo de nanogrid al cual podría llegar a ser una vivienda; la gestion de informacion adicional a la necesaria para cumplir la domótica resalta al a vista. En adición, estos componentes de generación y consumo eléctrico son la base para hablar sobre el impacto ambiental de las costumbres internas de los habitantes de un hogar, que resulta de vital interés para la optimización de los 4 ítems de calificación en el solar decathlon.
\vspace{0.5cm}\\
Finalizando, una propuesta integral de domótica y gestión inteligente de información como la que se presenta en este documento facilita la verificación y el cumplimiento de los ítems anteriormente mencionados, contemplando que el sistema quedara diseñado de tal manera que sea fácil la escalabilidad para aspectos como; seguridad, aseguramiento contra accidentes (incendio), viviendas de diferente tamaño, interfaz de voz y video.


\subsection{Objetivos}

\subsubsection{Objetivo general}
Implementar un sistema software para una vivienda del Solar Decathlon capaz de gestionar el control y la medición de variables físicas de manera local y remota.
\subsubsection{Objetivos especificos}
\begin{itemize}
	\item Diseñar una API que permita implementar interfaces gráficas, haciendo uso de imágenes personalizadas, para lograr una interacción amigable con el usuario.
	\item Implementar una aplicación software que funcione en el sitio y le permita al usuario gestionar la información de interés y controlar las cargas eléctricas.
	\item Implementar un aplicativo móvil que le permita al usuario tener la experiencia de interactuar con una vivienda inteligente.
\end{itemize}


\subsection{Lineamientos de desarrollo}
Debido a la naturaleza del proyecto, parte de los aspectos definidos al momento de establecer el anteproyecto fueron unas necesidades funcionales explicadas en el formato \ref{tab0}.

% Please add the following required packages to your document preamble:
% \usepackage[table,xcdraw]{xcolor}
% If you use beamer only pass "xcolor=table" option, i.e. \documentclass[xcolor=table]{beamer}

\begin{longtable}[c]{lllc}
\hline
\multicolumn{1}{|l|}{
\begin{tabular}[c]{@{}l@{}}
	\includegraphics[width=1cm]{figuras/uv_logo.png}
\end{tabular}
} 
& 
\multicolumn{2}{l|}{\cellcolor[HTML]{808080}{\color[HTML]{FFFFFF} \textbf{
\begin{tabular}[c]{@{}l@{}}Universidad del Valle\\--Implementación de un gestionado de recursos\\ para el hogar con inteligencia ambiental. ---\end{tabular}}}}
&
\multicolumn{1}{c|}{\textbf{\begin{tabular}[c]{@{}c@{}}Rev:\\ 003\end{tabular}}} \\ \hline
\endfirsthead
%
\endhead
%
\multicolumn{2}{|l|}{\textbf{\begin{tabular}[c]{@{}l@{}}Título:\\   ESPECIFICACIÓN DE \\ REQUERIMIENTOS \\ FUNCIONALES\end{tabular}}} & \multicolumn{1}{l|}{\textbf{\begin{tabular}[c]{@{}l@{}}Documento :\\ ERF-000\end{tabular}}} & \multicolumn{1}{c|}{\textbf{\begin{tabular}[c]{@{}c@{}}Página :\\ 1 de 1\end{tabular}}} \\ \hline
 &  &  & \multicolumn{1}{l}{} \\ \hline
\rowcolor[HTML]{808080} 
\multicolumn{4}{|c|}{\cellcolor[HTML]{808080}\textbf{REVISIÓN HISTÓRICA}} \\ \hline
\rowcolor[HTML]{808080} 
\multicolumn{1}{|l|}{\cellcolor[HTML]{808080}\textbf{Rev.}} & \multicolumn{1}{l|}{\cellcolor[HTML]{808080}\textbf{Descripción del Cambio}} & \multicolumn{1}{l|}{\cellcolor[HTML]{808080}\textbf{Autor}} & \multicolumn{1}{l|}{\cellcolor[HTML]{808080}\textbf{Fecha}} \\ \hline
\multicolumn{1}{|l|}{001} & \multicolumn{1}{l|}{Construcción del documento} & \multicolumn{1}{l|}{\begin{tabular}[c]{@{}l@{}}Juan David Ramirez\\ Villegas\end{tabular}} & \multicolumn{1}{l|}{18/03/2017} \\ \hline
\multicolumn{1}{|l|}{002} & \multicolumn{1}{l|}{Correcciones} & \multicolumn{1}{l|}{\begin{tabular}[c]{@{}l@{}}Juan David Ramírez\\ Villegas\end{tabular}} & \multicolumn{1}{l|}{15/11/2017} \\ \hline
\multicolumn{1}{|l|}{003} & \multicolumn{1}{l|}{Revisión} & \multicolumn{1}{l|}{\begin{tabular}[c]{@{}l@{}}Juan David Ramírez\\ Villegas\end{tabular}} & \multicolumn{1}{l|}{18/12/2017} \\ \hline
 &  &  & \multicolumn{1}{l}{} \\ \hline
\rowcolor[HTML]{808080} 
\multicolumn{1}{|l|}{\cellcolor[HTML]{808080}\textbf{Ref \#}} & \multicolumn{2}{l|}{\cellcolor[HTML]{808080}\textbf{Funciones}} & \multicolumn{1}{l|}{\cellcolor[HTML]{808080}\textbf{Categoría}} \\ \hline
\multicolumn{1}{|l|}{1.0} & \multicolumn{2}{l|}{USER APP (application móvil)} & \multicolumn{1}{c|}{\textbf{O}} \\ \hline
\multicolumn{1}{|l|}{1.1} & \multicolumn{2}{l|}{\begin{tabular}[c]{@{}l@{}}El usuario debe poder acceder a la \\ información medida en tiempo real.\end{tabular}} & \multicolumn{1}{c|}{\textbf{E}} \\ \hline
\multicolumn{1}{|l|}{1.2} & \multicolumn{2}{l|}{\begin{tabular}[c]{@{}l@{}}El usuario debe poder acceder a los datos \\ históricos recopilados en el mes y la \\ relación de sus gastos con ellos.\end{tabular}} & \multicolumn{1}{c|}{\textbf{O}} \\ \hline
\multicolumn{1}{|l|}{1.3} & \multicolumn{2}{l|}{\begin{tabular}[c]{@{}l@{}}El usuario debe poder cambiar los \\ parámetros de configuración escogidos\\  por defecto para la vivienda del\\  Solar Decathlon.\end{tabular}} & \multicolumn{1}{c|}{\textbf{E}} \\ \hline
\multicolumn{1}{|l|}{1.4} & \multicolumn{2}{l|}{\begin{tabular}[c]{@{}l@{}}El usuario debe recibir recomendaciones\\  para mejorar su entorno y su huella \\ hídrica cada cierto tiempo.\end{tabular}} & \multicolumn{1}{c|}{\textbf{E}} \\ \hline
\multicolumn{1}{|l|}{1.5} & \multicolumn{2}{l|}{\begin{tabular}[c]{@{}l@{}}La aplicación notificará al usuario\\ cuando la factura de energía o agua\\  esté vencida.\end{tabular}} & \multicolumn{1}{c|}{\textbf{O}} \\ \hline
\multicolumn{1}{|l|}{1.6} & \multicolumn{2}{l|}{\begin{tabular}[c]{@{}l@{}}El usuario debe ser capaz de ver un \\ índice del impacto ambiental de su\\  estilo de vida basado en datos \\ cualitativos y cuantitativos\end{tabular}} & \multicolumn{1}{c|}{\textbf{E}} \\ \hline
\multicolumn{1}{|l|}{1.7} & \multicolumn{2}{l|}{\begin{tabular}[c]{@{}l@{}}El sistema debe ser capaz de notificar \\ las pérdidas eléctricas, y por consiguiente\\  económicas, de un mal uso de los horarios\\  establecidos por defecto.\end{tabular}} & \multicolumn{1}{c|}{\textbf{O}} \\ \hline
\multicolumn{1}{|l|}{2.0} & \multicolumn{2}{l|}{HOUSE MANAGER (Aplicación en sitio)} & \multicolumn{1}{c|}{\textbf{E}} \\ \hline
\multicolumn{1}{|l|}{2.1} & \multicolumn{2}{l|}{\begin{tabular}[c]{@{}l@{}}El sistema debe mostrar gráficamente \\ la cantidad de agua y potencia consumida\\  durante el día contrastándolas con un \\ valor máximo recomendado.\end{tabular}} & \multicolumn{1}{c|}{\textbf{E}} \\ \hline
\multicolumn{1}{|l|}{2.2} & \multicolumn{2}{l|}{\begin{tabular}[c]{@{}l@{}}Por defecto, el uso de las cargas \\ eléctricas más significativas debe\\  programarse durante los picos de \\ generación en la casa. Estos horarios \\ podrán ser modificados bajo una \\ advertencia de uso no eficiente.\end{tabular}} & \multicolumn{1}{c|}{\textbf{E}} \\ \hline
\multicolumn{1}{|l|}{2.3} & \multicolumn{2}{l|}{\begin{tabular}[c]{@{}l@{}}El usuario debe poder acceder \\ a la información medida en tiempo real.\end{tabular}} & \multicolumn{1}{c|}{\textbf{E}} \\ \hline
\multicolumn{1}{|l|}{2.4} & \multicolumn{2}{l|}{\begin{tabular}[c]{@{}l@{}}El sistema debe poder comunicarse \\ con una base de datos que represente \\ todas las variables y el estado de los \\ circuitos de la vivienda.\end{tabular}} & \multicolumn{1}{c|}{\textbf{E}} \\ \hline
\caption{Requerimientos Funcionales del sistema}
\label{tab0}\\
\end{longtable}
La columna de la categoría indica si el requerimiento indicado es uno esencial (E) u opcional (O) para el desarrollo del sistema. Este documento será de gran importancia puesto que definió los lineamientos de desarrollo de todas las aplicaciones.




\subsection{Diseño Generalizado}

El sistema diseñado se diseño con las mejores condiciones de; modularidad, costo, tamaño y escalabilidad. Para cada uno de estas caracteristicas se ecogieron tecnologias y se desarrollaron funciones claves. 

Desde la perspectiva de la modularidad, se desarrollaron aplicaciones de frontend capaces de integrarse con 2 tipos de protocolos de comunicacion: mqtts y https, ademas, para acceder a los servicios de la nube se implemento el servicio de backend en la plataforma "Google Cloud Plataform" puesto que permite utilizar de la mejor manera todos los servicios de google: su asstente de voz, su interfaz de inteligencia artificial, sus servicios de autenticacion, su red privada de fibra optica que es de las mas rapidas y mayor cobertura, entre otros.

En cuanto al costó, se utilizaron tecnologias de backend totalmente escalables tipo serverless para ajustar los gastos a unicamente los necesarios, teniendo en cuenta que las pruebas a gran escala estuvieron lejos del alcance de este proyecto, aún asi se desarrollo toda la arquitectura pensando en la necesidad de optimizar en la mayor medida los gastos. Teniendo en cuenta que no se hicieron bruebas a gran escala los servicios de backend fueron posibles gracias a las coutas gratuitas del mayorista utilizado para desplegar el servicio (Google)

En cuanto al requerimiento de escalabilidad, la desicion de diseñar serverless juega el papel mas importante: Los servicios serverless le permitiran a la aplicacion permanecer funcional sin hacer ningun cambio inclusive si se tienen 4000 usuarios.

Para comprender mejor el producto, se puede observar en la figura \ref{diagramaGeneralizado} un diagrama generalizado de los componentes de software del sistema, donde principalmente los desarrollos hacen parte de 3 grandes bloques: el bloque de la nube, el bloque del dispositivo embebido (Raspberry), y el bloque del dispositivo movil (Android).

\begin{figure}[htbp]
	\centering
	\includegraphics[width=14cm]{figuras/GeneralDiagram.png}
	\caption{Diagrama generalizado de las estructuras de software}	
	\label{diagramaGeneralizado}
\end{figure}
	
