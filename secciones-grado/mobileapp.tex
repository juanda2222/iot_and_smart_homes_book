\customsection{User app}

Para el desarrollo de este sistema se decidió implementar la aplicación utilizando nuevas tecnologías de desarrollo multipltaforma para móviles como: React native, Flutter, Ionic. Lo anterior debido a que, dado que se desea proyectar la aplicacion a una escala mundial, es necesario desarrollar aplicativos para ambos sistemas operativos (Ios y Android), pero tener un equipo de desarrollo de ambas plataformas resulta mas costoso que tener un equipo de una tecnología híbrida.
\vspace{0.5cm}\\
Teniendo en cuenta lo anterior, los frameworks para el desarrollo de aplicaciones móviles híbridas más famosos en la actualidad son React Native, Flutter e Ionic.
\vspace{0.5cm}\\
Por una parte, React Native es un framework open source de Javasript creado y mantenido por Facebook Inc y posee 78,400 estrellas en el repositorio publico en Github \cite{mobilestadistics}. 
\vspace{0.5cm}\\
Por otro lado, Flutter es un framework para el lenguaje de programación Dart. Ambos Dart y Flutter son creados y mantenidos por Google y el repositorio de Flutter posee 68,000 estrellas en Github\cite{mobilestadistics}.
\vspace{0.5cm}\\
Finalmente, Ionic es un framework de Javascript creado y mantenido por la corporacion Drift Co. Su repositorio en Guithub tiene 39,100 estrellas.
\vspace{0.5cm}\\
Teniendo en cuenta lo anterior, se escogió el framework con mayor acogida por la comunidad de desarrolladores, es decir, React Native. En adición, la desición tambien tuvo en mente la necesidad alejarse de productos no estables, puesto que los frameworks de desarrollo móvil multi plataforma son muy nuevos y pueden existir bugs no reportados. Por ejemplo, Flutter, no tiene una versión estable soportada por Google.
\vspace{0.5cm}\\
React Native permite desarrollar aplicaciones nativas para Android y iOS a través de un sistema de traducción de componentes desde JavaScript al lenguaje nativo de la maquina (Java para Android y Swift para IOS). Por ende, las aplicaciones desarrolladas en este framework tienen un desempeño muy similar a las implementadas directamente en los lenguajes nativos correspondientes.
\vspace{0.5cm}\\
Además, una ventaja estratégica que tiene el desarrollar aplicaciones a través de este framework es que ambos desarrollos (para IOS y Android) quedan con las mismas interfaces gráficas, lo que le da al cliente la misma experiencia de usuario sin importar en que plataforma se encuentre.
\vspace{0.5cm}\\
Aunque desarrollar aplicaciones para Android y IOS con el mismo lenguaje de programación suena extremadamente ventajoso, hay limitaciones técnicas que serán mencionadas más adelante en este capítulo.

\subsection{Especificaciones técnicas}

Para poder realizar el aplicativo se utilizaron diferentes librerías de JavaScript creadas específicamente para el framework de React-Native. Las librerías utilizadas en el aplicativo fueron:

\begin{itemize}
	\item \textbf{React-native-google-signin:} Esta librería fue utilizada para obtener las llaves privadas y la información de usuario utilizando el servicio de autenticación de Google.
	\item \textbf{React-native-progress:} Con esta librería se lograron desarrollar los indicadores de progreso circulares para mostrar el índice de consumo eléctrico de agua y el coeficiente de sostenibilidad.
	\item \textbf{Jsrsasign:} Usando esta librería fue posible encriptar los JWT de autenticación a partir del protocolo de RSA de 256 bits. Esta librería fue de vital importancia puesto que poseía toda la compatibilidad necesaria para ser ejecutada con JavaScript puro, sin el uso de paquetes adicionales escritos en Java o Swift.
	\item \textbf{React-native-svg:} Esta librería fue utilizada para apoyar el desarrollo de componentes visuales adicionados a las gráficas.
	\item \textbf{React-native-svg-charts:} Usando esta librería fue posible integrar gráficas en el aplicativo móvil.
	\item \textbf{React-native-swiper:} Con esta librería se implementó la interfaz tipo deslizante (``swipe'') en las ventanas de introducción de la aplicación.
	\item \textbf{React-native-vector-icons:} Con esta librería fue posible utilizar los iconos nativos de Android y IOS como el botón de menú, de home, entre otros.
	\item \textbf{Toggle-switch-react-native:} Con esta librería se implementaron botones conmutadores (on off), utilizados en la escena de control.
	\item \textbf{React-navigation:} Esta librería fue utilizada para establecer el control de la navegación y el menú dentro de la aplicación como se puede ver en la figura \ref{fig_17}.
	\item \textbf{React-redux:} Usando esta librería fue posible comunicar de manera efectiva todos los objetos y componentes del sistema.
	\item \textbf{Fetch:} Con esta librería se estableció la comunicación basada en solicitudes Https con el backend.
\end{itemize}
\begin{figure}[htbp]
	\centerline{\includegraphics[width=6cm]{./figuras/mobile_menu.jpeg}}
	\caption{El menú desplegable con las aplicaciones actualmente desplegadas. Fuente: propia}
	\label{fig_17}
\end{figure}
Teniendo en cuenta que React\_native traduce todos sus componentes al lenguaje nativo del celular, es necesario entender que no todas las librerías de JavaScript se pueden usar para el framework, únicamente aquellas que tienen incluido el enlace de bajo nivel con ambos sistemas operativos. Por ende, las librerías con mayor compatibilidad son las que están desarrolladas exclusivamente con JavaScript puro. 
\vspace{0.5cm}\\
Con la anterior consideración en mente, en el proceso de desarrollo se presentó un problema por asumir incorrectamente que un ``snipet'' (Archivo o script que cumple una función en específico) era capaz de funcionar en el entorno de programación de React-Native. Este pequeño ``snipet'' fue desarrollado pensando en integrar las funcionalidades del protocolo de comunicación Mqtts a la aplicación; pero usaba librerías que estaban basadas en paquetes de bajo nivel equivocados, es decir, para computadores de uso comercial como Linux o Windows. Por ende, las funcionalidades de tiempo real fueron implementadas utilizando únicamente el protocolo de comunicación Https.
\vspace{0.5cm}\\
Para finalizar, de las librerías utilizadas, se puede observar que la mayoría son para mejorar la interfaz gráfica, mejorar la experiencia de usuario u obtener la misma vista entre sistemas operativos Android y Ios.

\subsection{Requerimientos Funcionales}

Debido a que durante el proceso de planeación del proyecto también se utilizó la tabla de requerimientos \ref{tab0} para definir las especificaciones de esta aplicación, y estos requerimientos guiaron el proceso de desarrollo, se explicará cómo se implementaron las especificaciones establecidas:

\begin{enumerate}
	\item \textsl{``El usuario debe poder acceder a la información medida en tiempo real:''} Para solucionar este requerimiento, en una primera etapa de desarrollo, se optó por utilizar una librería llamada ``react-native-mqtt''. Esta librería funcionaria como puente entre las librerías: ``Paho Mqtt Client' de Android y ``Mqtt Framework'' de IOS, pero debido al estado tan pionero de las tecnologías de desarrollo de React Native y el protocolo de comunicación Mqtt, la librería no era estable y muchas otras similares tampoco. 
	\vspace{0.5cm}\\
	Teniendo en cuenta lo anterior se implementó la comunicación en ambos sentidos a través de las solicitudes Https, por ende, para obtener los datos en tiempo real en la aplicación le realiza solicitudes al backend de tal forma que reciba los últimos datos guardados en el sistema de almacenamiento MySql.
	\vspace{0.5cm}\\
	Finalmente, para mostrar la información medida en tiempo real se implementó la escena de medición ``measure'' tal como se ve en la figura \ref{fig_15}, donde deslizándose hacia abajo se puede tener acceso a las gráficas con contenido de otros sensores.
	\begin{figure}[htbp]
		\centerline{\includegraphics[width=6cm]{./figuras/mobile_measure.jpeg}}
		\caption{Escena para la medición en tiempo real de las variables. Fuente: propia}
		\label{fig_15}
	\end{figure}
	\item  \textit{``El usuario debe ser capaz de cambiar los parámetros de configuraciones establecidos por defecto para la vivienda del solar Decathlon:''} Por defecto los parámetros de la vivienda del Solar Decathlon involucran unos horarios de uso de las cargas eléctricas, por ende, cuando el usuario realiza un cambio en la aplicación celular el sistema en sitio desactiva su configuración predeterminado. 
	\vspace{0.5cm}\\
	Finalmente, se implementó una escena de configuración. Esta escena, no se añadió ninguna entrada de usuario y se dejó para ilustrar la posibilidad de incluirlo en el futuro. La escena se puede observar en la figura \ref{fig_18}. En conclusión, no se cumplió este requerimiento.
	
	\begin{figure}[htbp]
		\centerline{\includegraphics[width=5cm]{./figuras/mobile_settings.jpeg}}
		\caption{Escena de configuraciones, actualmente no disponible. Fuente: propia}
		\label{fig_18}
	\end{figure}
	
	\item \textit{``El usuario debe poder acceder al histórico del mes y la relación de sus gastos con ellos:''} Para implementar este requerimiento se utilizaron barras circulares de progreso que indican la cantidad de energía o agua que ha consumido el usuario en el día, comparándola con un valor promedio. La escena desarrollada para solucionar este requerimiento se puede ver en la figura \ref{fig_13} donde el usuario puede deslizar horizontalmente el centro de la pantalla para revelar los otros indicadores.
	\begin{figure}[htbp]
		\centerline{\includegraphics[width=5cm]{./figuras/mobile_home.jpeg}}
		\caption{Escena para la visualización del consumo histórico. Fuente: propia}
		\label{fig_14}
	\end{figure}
	\item  \textit{``La aplicación notificará al usuario cuando la factura este vencida:''} para este requerimiento se implementó una función adicional en el aplicativo en sitio, por ende, el requerimiento se cumplió desde la aplicación de escritorio.
	
	\item \textit{``El usuario debe ser capaz de poder ver su impacto ambiental basado en datos cualitativos y cuantitativos:''} Teniendo en cuenta la figura \ref{fig_14} uno de los indicadores corresponde a un indicador de sostenibilidad que está basado en la huella de carbono del consumo de agua y el consumo eléctrico, pero el componente cualitativo fue incluido en la notificacion e-mail al fin del mes.
	
	\item \textit{``El sistema debe ser capaz de notificar las perdidas eléctricas y por consiguiente económicas de un mal uso de los horarios establecidos por defecto:''} Para implementar este requerimiento se desarrolló en el aplicativo en sitio la función de notificación automática. Es decir las notificaciones por perdidas no fueron incluidas en esta aplicación.
	
\end{enumerate}


\subsection{Funciones adicionales}

En adición a los requerimientos funcionales se añadieron características a la conceptualización de diseño original, lo anterior, para mejorar el funcionamiento de la plataforma y la experiencia de usuario. Estas funciones se pueden ver a continuación:

\begin{enumerate}
	\item \textbf{Pantalla de control:} Teniendo en cuenta un diseño con el mayor control y administración de los actuadores y sensores conectados al sistema, para esta versión de la aplicación, se implementó una interfaz de control que se puede observar en la figura \ref{fig_16}, donde el usuario tiene la posibilidad de modificar la configuración horaria programada por defecto para el Solar Decathlon. Una vez el usuario altera el estado de los circuitos de la vivienda a través de la ventana, El sistema de agenda local se desactiva por 24 horas para darle la prioridad a lo que defina el usuario.
	
	\begin{figure}[htbp]
		\centerline{\includegraphics[width=5cm]{./figuras/mobile_control.jpeg}}
		\caption{Escena para el control de los circuitos de la vivienda. Fuente: propia}
		\label{fig_16}
	\end{figure}
	
	\item \textbf{Ventana de introducción:} Esta función es más un componente decorativo, la escena le da la bienvenida al usuario, mejora la experiencia de usuario y le da status a la interfaz. Figura \ref{fig_12}.
	
	\begin{figure}[htbp]
		\centerline{\includegraphics[width=5cm]{./figuras/mobile_intro1.jpeg}}
		\caption{Escena de bienvenida 1. Fuente: propia}
		\label{fig_12}
	\end{figure}
	
	\item \textbf{Servicio de autenticación:} Finalmente, se implementó un servicio de autenticación utilizando la api de Google, utilizando su api de autenticación se desarrolló la interfaz para obtener la información del usuario como: correo, numero de teléfono, nombre completo y una llave encriptada con la que la app puede re autenticar el usuario en caso de ser necesario tal como se ve en la figura \ref{fig_13}.
	
	\begin{figure}[htbp]
		\centerline{\includegraphics[width=5cm]{./figuras/mobile_intro2.jpeg}}
		\caption{Escena de inicio de sesión. Fuente: propia}
		\label{fig_13}
	\end{figure}	
\end{enumerate}

